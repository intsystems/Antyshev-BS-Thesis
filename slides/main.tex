\documentclass{beamer}
\beamertemplatenavigationsymbolsempty
\usecolortheme{beaver}
\setbeamertemplate{blocks}[rounded=true, shadow=true]
\setbeamertemplate{footline}[page number]
%
\usepackage[utf8]{inputenc}
\usepackage[demo]{graphicx}
\usepackage[english]{babel}
\usepackage{amssymb,amsfonts,amsmath,mathtext}
\usepackage{subfig}
\usepackage[all]{xy} % xy package for diagrams
\usepackage{natbib}
\usepackage{array}
\usepackage{comment}
\usepackage{multicol}% many columns in slide
\usepackage{hyperref}% urls
\usepackage{hhline}%tables
% Your figures are here:
\graphicspath{ {fig/} {../fig/} }

\newcommand{\upperRomannumeral}[1]{\uppercase\expandafter{\romannumeral#1}}

%----------------------------------------------------------------------------------------------------------
\begin{comment}
\title[\hbox to 56mm{Feature generation}]{Feature generation for classification and forecasting problems}
\author[N.\,P.~Ivkin]{Nikita Ivkin}
\institute{Moscow Institute of Physics and Technology}
\date{\footnotesize
\par\smallskip\emph{Course:} My first scientific paper\par (Strijov's practice)/Group 874 %821, 813
\par\smallskip\emph{Expert:} I.\,F.~Anny
\par\smallskip\emph{Consultant:} I.\,O.~Gordeos
\par\bigskip\small 2021}
\end{comment}

\title[]{Gradient Sliding for SP with one composite}
\author{By Antyshev Tikhon \\$4^{\text{th}}$ year MIPT DCAM IS student}


\date[]{2023}

%---------------------------------------------------------------------------------------------------------
\begin{document}

\begin{frame}
\thispagestyle{empty}
\maketitle
\end{frame}


%----------------------------------------------------------------------------------------------------------

\begin{frame}{Establishing goals}
\begin{itemize}
    \item We consider following saddle point problem with one composite:
\begin{equation*}
    \min\limits_{x }\max\limits_{y} F(x, y) + f(x)
\end{equation*}
where functions $f(x)$ and $F(x,y)$ are correspondingly $L_f$ and $L_F$ Lipshitz continuous

\vspace{0.2in}

\item We also consider the sum of the two functions to be $\mu_x$ and $\mu_y$ strongly convex

%\item Federated Learning avoids centralized data-storing and instead %relies on clients for greater energy efficiency and privacy.
\end{itemize}
%\begin{block}{Project goal}
%The goal of the project is solving the problem using Local SGD and %compare it to the Federated Random Reshuffling.
%\end{block}

\end{frame}
%-----------------------------------------------------------------------------------------------------
\begin{frame}{Gradient Sliding}
    \begin{itemize} 
        \item The first step is to use the gradient sliding algorithm (\cite{KovBorGas2022}). 

        \vspace{0.2in}

        \item The oracle of $\nabla f(x)$ will be called $\tilde{\mathcal{O}}\left( \sqrt{\frac{L_f}{\mu_x}}\right)$

        \end{itemize}
        \vspace{0.2in}

\textbf{Problem:} we need to solve a suboptimization task.
    
\end{frame}
%-----------------------------------------------------------------------------------------------------
\begin{frame}{FOAM}
The gradient sliding produces the following suboptimization task:
\[\min\limits_{x }\max\limits_{y} \left(\langle \nabla f(x^k_g), x - x_g^k \rangle +  \Vert x-x^k_g \Vert^2 + F(x, y)\right)\]
\begin{itemize}
    \item The complexity of this task via FOAM (\cite{KovGas2022}) is $\tilde{\mathcal{O}}\left(\frac{L_F}{\sqrt{(L_f + \mu_x)\cdot \mu_y}}\right)$
\end{itemize}
\end{frame}



%----------------------------------------------------------------------------------------------------------

\begin{frame}{Final complexity}
The final oracle complexity for $\nabla F$ can now be calculated:
\[\tilde{\mathcal{O}}\left( \sqrt{\frac{L_f}{\mu_x}}\right) \cdot \tilde{\mathcal{O}}\left(\frac{L_F}{\sqrt{(L_f + \mu_x)\cdot \mu_y}}\right) = \tilde{\mathcal{O}}\left(\frac{L_f}{\mu_x\mu_y} \right)\]

\end{frame}



%----------------------------------------------------------------------------------------------------------

\begin{frame}{The Experiment }
The test is performed on a Bilinear Quadratic SP.
\[\min_x \max_y \left(\underbrace{p(x)} + <y, Ax> - g(y)\right)\]


\begin{figure}
\centering
\begin{minipage}{.5\textwidth}
  \centering
  \includegraphics[width=.99\linewidth]{bilin.png}
  \captionof{figure}{Visualized results}
  \label{fig:test1}
\end{minipage}%
\begin{minipage}{.5\textwidth}
  
\end{minipage}
\end{figure}




\end{frame}


\begin{frame}{Conclusion}
\begin{itemize}
\item A comparison with other methods is necessary

\item Run method on more SP tasks

\item The theory needs to be finished
\end{itemize}



\end{frame}





%----------------------------------------------------------------------------------------------------------

\begin{frame}{References}
\nocite{*}
{\small
\bibliographystyle{unsrtnat}
\bibliography{references.bib}
}
\end{frame}
%----------------------------------------------------------------------------------------------------------


%----------------------------------------------------------------------------------------------------------

%\begin{frame}{Research proposal}
%\nocite{*}
%{\small
%\bibliographystyle{unsrtnat}
%\bibliography{references.bib}
%}
%
%\end{frame}
%----------------------------------------------------------------------------------------------------------
\begin{comment}

\begin{frame}{Solution}

\begin{columns}[c]
\column{0.6\textwidth}
    Column 1
\column{0.4\textwidth}
    Column 2
\end{columns}
\end{frame}

\end{comment}

%----------------------------------------------------------------------------------------------------------
\begin{comment}
\begin{frame}{Conclusion}
    \begin{block}{Forecast with hierarchical aggregation of}
    \begin{itemize}
        \item types of freight in
        \item stations, regions, and roads,
        \item for a day, week, month, and quarter.
    \end{itemize}
    \end{block}
\end{frame}
\end{comment}
%----------------------------------------------------------------------------------------------------------
\end{document} 
